\documentclass[aspectratio=169]{beamer}
\usetheme[theme=blue,logo=logowithtextvi]{HUST} 
\DeclareUnicodeCharacter{221E}{\ensuremath{\infty}}

\usepackage[T5]{fontenc}
\usepackage[utf8]{inputenc}
\usepackage{amsmath}
\usepackage{amsfonts}
\usepackage{amssymb}
\usepackage{graphicx}
\usepackage{adjustbox}
\usepackage{xcolor}
\usepackage{tikz}
\usepackage{minted}

\usetikzlibrary{positioning,calc,decorations.pathreplacing,trees}
\definecolor{codeblue}{RGB}{0,90,200}   
\definecolor{codegold}{RGB}{210,160,0}
\definecolor{HUSTBlue}{RGB}{0,51,102}

% ===== Sudoku grid with row/col indices + 3x3 block braces =====
\newcommand{\SudokuIndexedGrid}[1][0.55]{%
	\begin{tikzpicture}[x=#1cm,y=#1cm]
		% --- thin grid ---
		\draw[step=1, black!45, thin] (0,0) grid (9,9);
		
		% --- thick 3x3 separators (blue-ish) ---
		\draw[very thick, HUSTBlue] (0,0) rectangle (9,9);
		\foreach \k in {3,6} {
			\draw[very thick, HUSTBlue] (\k,0) -- (\k,9);
			\draw[very thick, HUSTBlue] (0,\k) -- (9,\k);
		}
		
		% --- column indices 0..8 on top ---
		\foreach \c in {0,...,8} {
			\node[font=\small] at (\c+0.5, 9.35) {\c};
		}
		
		% --- row indices 0..8 on left (0 at top) ---
		\foreach \r in {0,...,8} {
			\node[font=\small] at (-0.35, 8.5-\r) {\r};
		}
		
		% --- braces for block columns (bottom): 0,1,2 ---
		\draw[decorate, decoration={brace, amplitude=4pt, mirror}, HUSTRed, thick]
		(0,-0.15) -- (3,-0.15) node[midway, yshift=-10pt, font=\large] {0};
		\draw[decorate, decoration={brace, amplitude=4pt, mirror}, HUSTRed, thick]
		(3,-0.15) -- (6,-0.15) node[midway, yshift=-10pt, font=\large] {1};
		\draw[decorate, decoration={brace, amplitude=4pt, mirror}, HUSTRed, thick]
		(6,-0.15) -- (9,-0.15) node[midway, yshift=-10pt, font=\large] {2};
		
		% --- braces for block rows (right): 0,1,2 (top to bottom) ---
		\draw[decorate, decoration={brace, amplitude=4pt}, HUSTRed, thick]
		(9.15,9) -- (9.15,6) node[midway, xshift=12pt, font=\large] {0};
		\draw[decorate, decoration={brace, amplitude=4pt}, HUSTRed, thick]
		(9.15,6) -- (9.15,3) node[midway, xshift=12pt, font=\large] {1};
		\draw[decorate, decoration={brace, amplitude=4pt}, HUSTRed, thick]
		(9.15,3) -- (9.15,0) node[midway, xshift=12pt, font=\large] {2};
	\end{tikzpicture}%
}


% ===== Sudoku solution grid (shaded 3x3 blocks + numbers) =====
\definecolor{SudokuShade}{gray}{0.85}
\newcommand{\SudokuSolvedGrid}[2][0.55]{%
	\begin{tikzpicture}[x=#1cm,y=#1cm]
		% --- shade blocks where (blockRow + blockCol) even ---
		\fill[SudokuShade] (0,6) rectangle (3,9); % top-left
		\fill[SudokuShade] (6,6) rectangle (9,9); % top-right
		\fill[SudokuShade] (3,3) rectangle (6,6); % center
		\fill[SudokuShade] (0,0) rectangle (3,3); % bottom-left
		\fill[SudokuShade] (6,0) rectangle (9,3); % bottom-right
		
		% --- grid lines ---
		\draw[step=1, black, thin] (0,0) grid (9,9);
		\draw[very thick] (0,0) rectangle (9,9);
		\foreach \k in {3,6} {
			\draw[very thick] (\k,0) -- (\k,9);
			\draw[very thick] (0,\k) -- (9,\k);
		}
		
		% --- numbers: #2 is a list of 9 rows, each row is 9 numbers ---
		\foreach \row [count=\r from 0] in {#2} {
			\foreach \val [count=\c from 0] in \row {
				\node[font=\large\bfseries] at (\c+0.5, 8.5-\r) {\val};
			}
		}
	\end{tikzpicture}%
}

% ===== Backtracking tree (vector, manual coordinates) =====
\newcommand{\TryBacktrackingTree}[1][1.0]{%
	\begin{tikzpicture}[
		x=0.95cm, y=1.0cm, % <-- muốn "khít" ngang hơn: giảm x xuống 0.90cm, 0.85cm...
		scale=#1, every node/.style={transform shape},
		treenode/.style={
			circle, fill=HUSTBlue, draw=black, line width=0.9pt,
			minimum size=7mm, inner sep=0pt
		},
		edge/.style={draw=black, line width=0.9pt},
		lab/.style={font=\small}
		]
		% ---- ROOT + level 1 ----
		\node[treenode, label={[lab]above:Try(1)}] (r) at (0, 0) {};
		\node[treenode, label={[lab]above left:Try(2)}] (A) at (-3.6, -1.6) {};
		\node[treenode, label={[lab]above:Try(2)}]      (B) at (-0.4, -1.75) {};
		\node[treenode, label={[lab]above:Try(2)}]      (C) at ( 3.9, -1.75) {};
		
		\draw[edge] (r)--(A);
		\draw[edge] (r)--(B);
		\draw[edge] (r)--(C);
		
		% Dấu "..." ở tầng 1 (giữa nhánh giữa và nhánh phải) như ảnh
		\node[lab] at (1.55, -2.05) {$\cdots$};
		
		% ---- Left subtree (A) ----
		\node[treenode, label={[lab]above left:Try(3)}]  (A1) at (-5.0, -3.2) {};
		\node[treenode, label={[lab]above right:Try(3)}] (A2) at (-2.4, -3.2) {};
		\draw[edge] (A)--(A1);
		\draw[edge] (A)--(A2);
		
		\node[treenode] (A11) at (-5.7, -5.4) {};
		\node[treenode] (A12) at (-4.3, -5.4) {};
		\draw[edge] (A1)--(A11);
		\draw[edge] (A1)--(A12);
		
		% Dấu "..." ở vùng trống dưới-trái (như ảnh)
		\node[lab] at (-2.05, -4.35) {$\cdots$};
		
		% ---- Middle subtree (B) ----
		\node[treenode] (B1) at (-1.55, -3.45) {}; % lá trống như ảnh
		\node[treenode, label={[lab]above:Try(3)}] (B2) at ( 0.95, -3.45) {};
		\draw[edge] (B)--(B1);
		\draw[edge] (B)--(B2);
		
		\node[treenode] (B21) at ( 0.10, -5.45) {};
		\node[treenode] (B22) at ( 1.85, -5.45) {};
		\draw[edge] (B2)--(B21);
		\draw[edge] (B2)--(B22);
		
		% ---- Right subtree (C) ----
		\node[treenode] (C1) at ( 2.65, -3.55) {};
		\node[treenode] (C2) at ( 5.05, -3.55) {};
		\draw[edge] (C)--(C1);
		\draw[edge] (C)--(C2);
		
		% Dấu "..." giữa 2 lá bên phải như ảnh
		\node[lab] at (3.85, -3.55) {$\cdots$};
		
	\end{tikzpicture}%
}

% ===== Binary string tree n=3 (SAFE macro: no tikzpicture environment) =====
% ===== Binary string tree n=3 (vector TikZ) =====
% (Edge: chỉ có nhãn 0/1 màu đỏ) ; (Try(k): nhãn node màu đen)
\newcommand{\BinaryStringTreeNThree}[1][0.90]{%
	{%
		\centering
		\scalebox{#1}{%
			\begin{tikzpicture}[
				grow=down,
				level distance=16mm,
				level 1/.style={sibling distance=78mm},
				level 2/.style={sibling distance=46mm},
				level 3/.style={sibling distance=20mm},
				bstnode/.style={circle, draw=black, line width=0.8pt, fill=HUSTRed, minimum size=7mm, inner sep=0pt},
				bstedge/.style={draw=HUSTRed, line width=2.0pt},
				bstbit/.style={text=HUSTRed, font=\bfseries\Large},
				bsttry/.style={text=black, font=\small},
				bstpref/.style={text=black, font=\bfseries\Large},
				bstleaf/.style={text=black, font=\bfseries\Large},
				edge from parent/.style={bstedge}
				]
				
				% --- structure (KHÔNG gắn nhãn Try lên edge) ---
				\node[bstnode] (r) {}
				child { node[bstnode] (n0) {}
					child { node[bstnode] (n00) {}
						child { node[bstnode] (n000) {} }
						child { node[bstnode] (n001) {} }
					}
					child { node[bstnode] (n01) {}
						child { node[bstnode] (n010) {} }
						child { node[bstnode] (n011) {} }
					}
				}
				child { node[bstnode] (n1) {}
					child { node[bstnode] (n10) {}
						child { node[bstnode] (n100) {} }
						child { node[bstnode] (n101) {} }
					}
					child { node[bstnode] (n11) {}
						child { node[bstnode] (n110) {} }
						child { node[bstnode] (n111) {} }
					}
				};
				
				% --- Try labels (đặt gần node, tránh đè lên edge) ---
				\node[bsttry, yshift=2.5mm] at (r.north) {Try(1)};
				\node[bstpref, yshift=10mm] at (r.north) {()};
				
				\node[bsttry, xshift=6mm, yshift=2mm] at (n0.north) {Try(2)};
				\node[bsttry, xshift=-6mm, yshift=2mm] at (n1.north) {Try(2)};
				
				\node[bsttry, xshift=6mm, yshift=2mm] at (n00.north) {Try(3)};
				\node[bsttry, xshift=6mm, yshift=2mm] at (n01.north) {Try(3)};
				\node[bsttry, xshift=6mm, yshift=2mm] at (n10.north) {Try(3)};
				\node[bsttry, xshift=6mm, yshift=2mm] at (n11.north) {Try(3)};
				
				% --- prefix labels (0),(1),(00),(01),(10),(11) ---
				\node[bstpref, xshift=-8mm, yshift=4mm] at (n0.north) {(0)};
				\node[bstpref, xshift= 8mm, yshift=4mm] at (n1.north) {(1)};
				
				\node[bstpref, xshift=-18mm, yshift=2mm] at (n00.west) {(00)};
				\node[bstpref, xshift= 18mm, yshift=2mm] at (n01.east) {(01)};
				\node[bstpref, xshift=-20mm, yshift=2mm] at (n10.west) {(10)};
				\node[bstpref, xshift= 15mm, yshift=2mm] at (n11.east) {(11)};
				
				% --- leaf labels ---
				\node[bstleaf, yshift=-10mm] at (n000.south) {(000)};
				\node[bstleaf, yshift=-10mm] at (n001.south) {(001)};
				\node[bstleaf, yshift=-10mm] at (n010.south) {(010)};
				\node[bstleaf, yshift=-10mm] at (n011.south) {(011)};
				\node[bstleaf, yshift=-10mm] at (n100.south) {(100)};
				\node[bstleaf, yshift=-10mm] at (n101.south) {(101)};
				\node[bstleaf, yshift=-10mm] at (n110.south) {(110)};
				\node[bstleaf, yshift=-10mm] at (n111.south) {(111)};
				
				% --- edge bit labels (0/1): đặt trên edge, gần phía trên như ảnh mẫu ---
				\path (r)   -- (n0)  node[bstbit, midway, above] {0};
				\path (r)   -- (n1)  node[bstbit, midway, above] {1};
				
				\path (n0)  -- (n00) node[bstbit, midway, above] {0};
				\path (n0)  -- (n01) node[bstbit, midway, above] {1};
				
				\path (n1)  -- (n10) node[bstbit, midway, above] {0};
				\path (n1)  -- (n11) node[bstbit, midway, above] {1};
				
				\path (n00) -- (n000) node[bstbit, midway, above] {0};
				\path (n00) -- (n001) node[bstbit, midway, above] {1};
				
				\path (n01) -- (n010) node[bstbit, midway, above] {0};
				\path (n01) -- (n011) node[bstbit, midway, above] {1};
				
				\path (n10) -- (n100) node[bstbit, midway, above] {0};
				\path (n10) -- (n101) node[bstbit, midway, above] {1};
				
				\path (n11) -- (n110) node[bstbit, midway, above] {0};
				\path (n11) -- (n111) node[bstbit, midway, above] {1};
				
			\end{tikzpicture}%
		}%
		\par
	}%
}









\setminted{
	breaklines=false,
	autogobble=false,
	obeytabs=true,
	tabsize=2,
	linenos=true,
	showspaces=false,
	space=~,
	baselinestretch=1,
	fontsize=\normalsize,
	rulecolor=\color{black}
}

\newcommand{\placecontent}[4]{%
  \tikz[remember picture,overlay]
    \node[anchor=north west]
      at ([xshift=#1,yshift=-#2]current page.north west)
      {\parbox{#3}{#4}};
}

\graphicspath{{./week 3 resources/}}

\title{\huge CẤU TRÚC DỮ LIỆU VÀ GIẢI THUẬT}
\author{SoICT - HUST}
\date{}

\begin{document}

% 2 slides đầu tiên:
\HUSTInsertBrandSlide
\HUSTInsertThemeSlide

% Slide tiêu đề
{\HUSTUseBackground{onelove.pdf}
\begin{frame}
  \ifdefstring{\insertaspectratio}{169}{
    \HUSTCornerImage[0.14]{assets/logo/soict_vi_h.pdf}
    \placecontent{0.5cm}{0.33\paperheight}{0.85\paperwidth}{
        \color{\HUSTFrameTitleTextColor}\bfseries\fontsize{22pt}{30pt}\selectfont
        \inserttitle
    }
    \placecontent{0.5cm}{0.50\paperheight}{0.8\paperwidth}{
        \color{\HUSTFrameTitleTextColor}\fontsize{14pt}{14pt}\selectfont
        %Bài học
        \textbf{\large TUẦN 3: ĐỆ QUY QUAY LUI}\\
    }
  }{}
\end{frame}
}

% Outline
\AtBeginSection[]
{
    \begin{frame}<beamer>
        \frametitle{NỘI DUNG}
        \tableofcontents[currentsection]
    \end{frame}
}

%Nội dung chính trong slides

\section{Sơ đồ chung đệ quy quay lui}
\begin{frame}[t]{SƠ ĐỒ CHUNG} 
	\setlength{\leftmargini}{-1.2em}
	
	\begin{itemize}
		\item Bài toán liệt kê tổ hợp: Liệt kê các bộ $x = (x[1], x[2], \dots, x[k], x[k+1], \dots, x[n])$ với $x[i] \in A_i$, $i = 1,2,\dots,n$ và thỏa mãn tập các ràng buộc $P$ cho trước.
		\vspace*{0.5cm}
		
		\item Ví dụ:
		\begin{itemize}
			\item ``Bài toán liệt kê xâu nhị phân độ dài $n$'' dẫn về liệt kê các bộ $x = (x[1], x[2], \dots, x[k], x[k+1], \dots, x[n])$
			với $x[i] \in \{0,1\}$, $i = 1,2,\dots,n$. 
			\vspace*{0.5cm}
			
			\item ``Bài toán liệt kê xâu nhị phân độ dài $n$ \textit{có số lượng bit $0$ là một số chẵn}'' dẫn về liệt kê các bộ 
			$x = (x[1], x[2], \dots, x[k], x[k+1], \dots, x[n])$ với $x[i] \in \{0,1\}$, $i = 1,2,\dots,n$ và thỏa mãn ràng buộc:
			số lượng phần tử $x[i] = 0$ với $i = 1,2,\dots,n$ là một số chẵn.
		\end{itemize}
		\vspace*{0.5cm}
		
		\item Thuật toán quay lui cho phép giải các bài toán liệt kê tổ hợp.
		Có hai cách cài đặt thuật toán quay lui: đệ quy hoặc không đệ quy.
	\end{itemize}
\end{frame}




\begin{frame}[t,fragile]{SƠ ĐỒ CHUNG}
	\small
	\setlength{\leftmargini}{-1.2em}
	
	%=========== BULET ĐẦU: FULL WIDTH ===========
	\begin{itemize}
		\item Bài toán liệt kê tổ hợp: Liệt kê các bộ
		$x = (x[1], x[2], \dots, x[k], x[k+1], \dots, x[n])$
		với $x[i] \in A_i$, $i = 1, 2, \dots, n$ và thỏa mãn
		tập các ràng buộc $P$ cho trước.
	\end{itemize}
	
	\vspace{0.4em}
	
	%=========== PHẦN CÒN LẠI: CHIA 2 CỘT ===========
	\begin{columns}[T,onlytextwidth]
		
		%----- Cột trái: các bullet còn lại -----
		\begin{column}{0.47\textwidth}
			\setlength{\leftmargini}{-1.2em}
			\begin{itemize}
				\item Lệnh gọi để thực hiện thuật toán đệ
				quy quay lui là: {\color{HUSTRed}\texttt{Try(1);}}
				
				\item Nếu chỉ cần tìm một lời giải thì cần tìm cách
				chấm dứt các thủ tục gọi đệ qui lồng nhau
				sinh bởi lệnh gọi Try(1) sau khi ghi nhận
				được lời giải đầu tiên.
				
				\item Nếu kết thúc thuật toán mà ta không thu được
				một lời giải nào thì điều đó có nghĩa là bài
				toán không có lời giải.
			\end{itemize}
		\end{column}
		
		%----- Cột phải: khung code Try(k) -----
		\begin{column}{0.53\textwidth}
			\vspace{0.4em} % chỉnh cho nó thẳng hàng đẹp
			\centering
			\begin{minipage}[t]{0.95\textwidth}
				\begin{minted}[
					fontsize=\scriptsize,
					frame=single,
					framesep=1mm,
					numbers=none,
					autogobble=true,
					escapeinside=||,
					baselinestretch=1.2,
					formatcom=\color{HUSTBlue} 
					]{text}
					Try(k) {|{\textcolor{codegold}{// thử các giá trị có thể gán cho x[k]}}|
						for v in candidates(k) do {
							if (check(v,k)) then {
								x[k] = v;
								[Update the data structure D]
								if (k == n) then solution();
								else Try(k+1);
								[Recover the data structure D]
							}
						}
					}
				\end{minted}
			\end{minipage}
		\end{column}
	\end{columns}
\end{frame}




\begin{frame}[t,fragile]{SƠ ĐỒ CHUNG}
	\small
	\setlength{\leftmargini}{-1.2em}
	
	%=========== BULET ĐẦU: FULL WIDTH ===========
	\begin{itemize}
		\item Bài toán liệt kê tổ hợp: Liệt kê các bộ
		$x = (x[1], x[2], \dots, x[k], x[k+1], \dots, x[n])$
		với $x[i] \in A_i$, $i = 1, 2, \dots, n$ và thỏa mãn
		tập các ràng buộc $P$ cho trước.
	\end{itemize}
	
	\vspace{0.3em}
	
	%=========== PHẦN CÒN LẠI: CHIA 2 CỘT ===========
	\begin{columns}[T,onlytextwidth]
		
		%----- Cột trái: bullet + hình cây -----
		\begin{column}{0.47\textwidth}
			\setlength{\leftmargini}{-1.2em}
			\begin{itemize}
				\item Lệnh gọi để thực hiện thuật toán đệ
				quy quay lui là:
				{\color{HUSTRed}\texttt{Try(1);}}
			\end{itemize}
			
		
			
			\TryBacktrackingTree[0.55]  
			 
			
		\end{column}
		
		%----- Cột phải: khung code Try(k) -----
		\begin{column}{0.51\textwidth}
			\vspace{1cm}
			\centering
			\begin{minipage}[t]{0.98\textwidth}
				\begin{minted}[
					fontsize=\scriptsize,
					frame=single,
					framesep=1mm,
					numbers=none,
					autogobble=true,
					baselinestretch=1.1,
					escapeinside=||,
					formatcom=\color{HUSTBlue} % code màu xanh
					]{text}
					Try(k) {|{\textcolor{orange}{// thử các giá trị có thể gán cho x[k]}}|
						for v in candidates(k) do {
							if (check(v,k)) then {
								x[k] = v;
								[Update the data structure D]
								if (k == n) then solution();
								else Try(k+1);
								[Recover the data structure D]
							}
						}
					}
				\end{minted}
			\end{minipage}
		\end{column}
		
	\end{columns}
\end{frame}

\section{Bài toán liệt kê xâu nhị phân}
\begin{frame}[t]{BÀI TOÁN LIỆT KÊ XÂU NHỊ PHÂN}
	\setlength{\leftmargini}{-1.2em}
	
	\begin{itemize}
		\item Cho số nguyên dương $n \ge 1$. Hãy liệt kê tất cả các xâu nhị phân
		độ dài $n$ theo thứ tự từ điển.
		
		\item Ví dụ: $n = 3$, ta có các xâu nhị phân độ dài $3$ cần liệt kê theo
		thứ tự như sau:
		\begin{itemize}
			\item 000
			\item 001
			\item 010
			\item 011
			\item 100
			\item 101
			\item 110
			\item 111
		\end{itemize}
	\end{itemize}
\end{frame}


\begin{frame}[t,fragile]{BÀI TOÁN LIỆT KÊ XÂU NHỊ PHÂN}
	\small
	\setlength{\leftmargini}{-1.2em}
	
	%============ PHẦN CHỮ (FULL WIDTH) ============
	\begin{itemize}
		\item Cho số nguyên dương $n \ge 1$. Hãy liệt kê tất cả các xâu nhị phân
		độ dài $n$ theo thứ tự từ điển.
		\item Biểu diễn lời giải: mỗi xâu nhị phân được biểu diễn bởi mảng
		$x = (x[1], x[2], \ldots, x[n])$, trong đó $x[k] \in \{0,1\}$ là bít thứ
		$k$ trong xâu nhị phân.
	\end{itemize}
	
	\vspace{0.3cm}
	
	%===== HAI KHUNG CODE + CỘT CHỮ + MŨI TÊN =====
	\begin{tikzpicture}
		%--- Khung code bên trái (pseudo-code) ---
		\node (L) [inner sep=0pt, anchor=north west] at (0,0)
		{%
			\begin{minipage}[t]{0.40\textwidth}
				\begin{minted}[
					fontsize=\tiny,
					frame=single,
					framesep=1mm,
					xleftmargin=-20pt,
					numbers=none,
					autogobble=true,
					baselinestretch=1.5,
					escapeinside=|| % để chèn \color
					]{text}
					|{\color{codeblue}Try(k) \{}||{\color{codegold}// thử các giá trị có thể gán cho x[k]}|
						|{\color{codeblue}    for v in candidates(k) do \{}|
							|{\color{codeblue}        if (check(v,k)) then \{}|
								|{\color{codeblue}            x[k] = v;}|
								|{\color{codeblue}            [Update the data structure D]}|
								|{\color{codeblue}            if (k == n) then solution();}|
								|{\color{codeblue}            else Try(k+1); }|
								|{\color{codeblue}            [Recover the data structure D]}|
							|{\color{codeblue}        \}}|
						|{\color{codeblue}    \}}|
					|{\color{codeblue}\}}|
				\end{minted}
			\end{minipage}
		};
		
		%--- Khung code bên phải: void Try(int k) ---
		\node (Rtop) [inner sep=0pt, anchor=north east] at (\textwidth,0)
		{%
			\begin{minipage}[t]{0.40\textwidth}
				\begin{minted}[
					fontsize=\tiny,
					frame=single,
					framesep=1mm,
					xrightmargin=-18pt,
					baselinestretch=1.2,
					numbers=none,
					autogobble=true
					]{c}
					void Try(int k){
						for (int v = 0; v <= 1; v++){
							if (check(v,k)){
								x[k] = v;
								if (k == n) solution();
								else Try(k + 1);
							}
						}
					}
				\end{minted}
			\end{minipage}
		};
		
		%--- Khung code bên phải phía dưới: int check(...) ---
		\node (Rbot) [inner sep=0pt, anchor=north east]
		at ([yshift=-0.1cm]Rtop.south east)
		{%
			\begin{minipage}[t]{0.40\textwidth}
				\begin{minted}[
					fontsize=\tiny,
					frame=single,
					framesep=1mm,
					xrightmargin=-18pt,
					numbers=none,
					baselinestretch=1.5,
					autogobble=true
					]{c}
					int check(int v, int k){
						return 1;
					}
				\end{minted}
			\end{minipage}
		};
		
		%--- Mũi tên + chữ "Cần xác định" ---
		\coordinate (Astart) at ([yshift=-0.2cm,xshift=0.1cm]L.east);
		\coordinate (Aend)   at ([yshift=-0.87cm,xshift=-0.2cm]Rtop.west);
		\draw[->] (Astart) -- (Aend);
		
		\node[anchor=south west] at ($(Astart)+(0.2,0.0)$)
		{\scriptsize
			\begin{tabular}{@{}l@{}}
				Cần xác định: \\
				\texttt{candidates(k)} \\
				\texttt{check(v,k)}
			\end{tabular}
		};
	\end{tikzpicture}
\end{frame}



\begin{frame}[t,fragile]{BÀI TOÁN LIỆT KÊ XÂU NHỊ PHÂN}
	\small
	
	\begin{columns}[T,onlytextwidth]
		%================= Cột trái: khung code =================
		\begin{column}{0.32\textwidth}
			\hspace*{-0.4cm}% dịch cả khung sang trái 0.3cm
			\begin{minipage}[t]{\dimexpr\linewidth+0.3cm\relax}
				\begin{minted}[
					fontsize=\scriptsize,
					frame=single,
					framesep=1mm,
					numbers=none,
					autogobble=true,
					baselinestretch=1.5,
					formatcom=\color{HUSTBlue} % code màu xanh
					]{c}
					void Try(int k){
						for (int v = 0; v <= 1; v++){
							x[k] = v;
							if (k == n) solution();
							else Try(k + 1);
						}
					}
				\end{minted}
			\end{minipage}
		\end{column}
		
		%================= Cột phải: tiêu đề + hình cây =================
		\begin{column}{0.68\textwidth}
			\centering
			Cây liệt kê các xâu nhị phân độ dài 3
			\vspace*{1.5cm}
			\BinaryStringTreeNThree[0.59] 
		\end{column}
	\end{columns}
\end{frame}

\section{Bài toán liệt kê hoán vị}

\begin{frame}[t]{BÀI TOÁN LIỆT KÊ HOÁN VỊ}
	\setlength{\leftmargini}{-1.2em}
	
	\begin{itemize}
		\item Cho số nguyên dương $n \ge 1$. Hãy liệt kê tất cả các hoán vị
		của $n$ số $1, 2, \ldots, n$ theo thứ tự từ điển.
		
		\item Ví dụ: $n = 3$, ta có các hoán vị của $1, 2, 3$ theo thứ tự
		từ điển như sau:
		\begin{itemize}
			\item $(1, 2, 3)$
			\item $(1, 3, 2)$
			\item $(2, 1, 3)$
			\item $(2, 3, 1)$
			\item $(3, 1, 2)$
			\item $(3, 2, 1)$
		\end{itemize}
	\end{itemize}
\end{frame}

\begin{frame}[t,fragile]{BÀI TOÁN LIỆT KÊ HOÁN VỊ}
	\small
	\setlength{\leftmargini}{-1.2em}
	
	%================= PHẦN CHỮ (FULL WIDTH) =================
	\begin{itemize}
		\item Biểu diễn lời giải: mỗi hoán vị $n$ phần tử được biểu diễn bởi
		mảng $x = (x[1], x[2], \ldots, x[n])$ trong đó:
		\begin{itemize}
			\item $x[k] \in \{1,2,\ldots,n\}$ là phần tử thứ $k$ trong hoán vị;
			\item $x[k] \neq x[1], x[2], \ldots, x[k-1], x[k+1], \ldots, x[n]$.
		\end{itemize}
	\end{itemize}
	
	\vspace{0.3cm}
	
	%===== HAI KHUNG CODE + CỘT CHỮ + MŨI TÊN =====
	\begin{tikzpicture}
		%--- Khung code bên trái (pseudo-code) ---
		\node (L) [inner sep=0pt, anchor=north west] at (0,0)
		{%
			\begin{minipage}[t]{0.40\textwidth}
				\begin{minted}[
					fontsize=\tiny,
					frame=single,
					framesep=1mm,
					xleftmargin=-20pt,
					numbers=none,
					autogobble=true,
					baselinestretch=1.3,
					escapeinside=|| % để chèn \color
					]{text}
					|{\color{codeblue}try(k) \{}||{\color{codegold}// thử các giá trị có thể gán cho x[k]}|
						|{\color{codeblue}    for v in candidates(k) do \{}|
							|{\color{codeblue}        if (check(v,k)) then \{}|
								|{\color{codeblue}            x[k] = v;}|
								|{\color{codeblue}            [Update the data structure D]}|
								|{\color{codeblue}            if (k == n) then solution();}|
								|{\color{codeblue}            else try(k+1); }|
								|{\color{codeblue}            [Recover the data structure D]}|
							|{\color{codeblue}        \}}|
						|{\color{codeblue}    \}}|
					|{\color{codeblue}\}}|
				\end{minted}
			\end{minipage}
		};
		
		%--- Khung code bên phải (C/C++) ---
		\node (R) [inner sep=0pt, anchor=north east] at (\textwidth,0)
		{%
			\begin{minipage}[t]{0.40\textwidth}
				\begin{minted}[
					fontsize=\tiny,
					frame=single,
					framesep=1mm,
					xrightmargin=7pt,
					numbers=none,
					autogobble=true,
					escapeinside=||,
					baselinestretch=1.5,
					formatcom=\color{HUSTBlue}
					]{c}
					void Try(int k){
						for (int v = 1; v <= n; v++){
							|{\color{HUSTRed}        if (check(v, k)) \{}|
							x[k] = v;
							if (k == n) solution();
							else Try(k + 1);
						}
					}
				}
			\end{minted}
		\end{minipage}
	};
	
	%--- Mũi tên + chữ "Cần xác định" ---
	\coordinate (Astart) at ([yshift=-0.2cm,xshift=0.1cm]L.east);
	\coordinate (Aend)   at ([yshift=-0.29cm,xshift=-0.2cm]R.west);
	\draw[->] (Astart) -- (Aend);
	
	\node[anchor=south west] at ($(Astart)+(0.2,0.0)$)
	{\scriptsize
		\begin{tabular}{@{}l@{}}
			Cần xác định: \\
			\texttt{candidates(k)} \\
			\texttt{check(v, k)}
		\end{tabular}
	};
\end{tikzpicture}
\end{frame}

\begin{frame}[t,fragile]{BÀI TOÁN LIỆT KÊ HOÁN VỊ}
	\small
	\setlength{\columnsep}{0.1cm} % cứ để nhỏ
	
	\begin{columns}[T,onlytextwidth]
		
		%=========== Cột trái ===========
		\begin{column}{0.40\textwidth}
			\raggedright
			% đẩy khung code vào giữa 1 chút
			\hspace*{0.2\textwidth}%
			\begin{minipage}[t]{\dimexpr\linewidth-0.05\textwidth\relax}
				\begin{minted}[
					fontsize=\scriptsize,
					frame=single,
					framesep=2mm,
					xleftmargin=-10pt,
					numbers=none,
					autogobble=true,
					escapeinside=||,
					formatcom=\color{codeblue}
					]{text}
					#include <stdio.h>
					int n;
					int x[100];
					
					void solution(){
						for (int k = 1; k <= n; k++)
						printf("%d ", x[k]);
						printf("\n");
					}
					
					int check(int v, int k) {
						for (int i = 1; i <= k-1; i++)
						if (x[i] == v) return 0;
						return 1;
					}
				\end{minted}
			\end{minipage}
		\end{column}
		
		%=========== Cột phải ===========
		\begin{column}{0.40\textwidth}
			\raggedright
			% đẩy khung code vào giữa theo chiều ngược lại
			\hspace*{-0.2\textwidth}%
			\begin{minipage}[t]{\dimexpr\linewidth+0.05\textwidth\relax}
				\begin{minted}[
					fontsize=\scriptsize,
					frame=single,
					framesep=2mm,
					xleftmargin=-10pt,
					numbers=none,
					autogobble=true,
					escapeinside=||,
					formatcom=\color{codeblue}
					]{text}
					void Try(int k){
						for (int v = 1; v <= n; v++){
							|{\color{HUSTRed}        if (check(v, k)) \{}|
							x[k] = v;
							if (k == n) solution();
							else Try(k + 1);
						}
					}
				}
				
				int main(){
					scanf("%d", &n);
					Try(1);
				}
			\end{minted}
		\end{minipage}
	\end{column}
	
\end{columns}
\end{frame}

\begin{frame}[t,fragile]{BÀI TOÁN LIỆT KÊ HOÁN VỊ}
	\small
	\setlength{\leftmargini}{-1.2em}
	
	%================= PHẦN CHỮ TRÊN CÙNG =================
	\begin{itemize}
		\item Kỹ thuật đánh dấu
		\begin{itemize}
			\item \texttt{used[v] = 1}: v đã xuất hiện
			\item \texttt{used[v] = 0}: v chưa xuất hiện
		\end{itemize}
	\end{itemize}
	
	\vspace{0.25cm}
	
	%================= HAI KHUNG CODE + MŨI TÊN =================
	\centering
	\begin{tikzpicture}
		%--- Khung trái: pseudo-code với 2 ô viền đỏ ---
		\node (L) [inner sep=0pt, anchor=north west] at (0,0)
		{%
			\begin{minipage}[t]{0.42\textwidth}
				\begin{minted}[
					fontsize=\tiny,
					frame=single,
					framesep=1mm,
					xleftmargin=-18pt,
					numbers=none,
					autogobble=true,
					escapeinside=||,
					baselinestretch=1.5,
					formatcom=\color{codeblue}
					]{text}
					try(k){|{\color{codegold}// thử các giá trị có thể gán cho x[k]}|
						for v in candidates(k) do {
							|{\fcolorbox{red}{white}{\ttfamily\color{red}if (check(v,k)) then \{}}|
							x[k] = v;
							[Update the data structure D]
							|{\fcolorbox{red}{white}{\ttfamily\color{red}if (k == n) then solution();}}|
							else try(k+1);
							[Recover the data structure D]
						}
					}
				\end{minted}
			\end{minipage}
		};
		
		%--- Khung phải: code C với used[v] tô đỏ ---
		\node (R) [inner sep=0pt, anchor=north east] at (\textwidth,0)
		{%
			\begin{minipage}[t]{0.42\textwidth}
				\begin{minted}[
					fontsize=\tiny,
					frame=single,
					framesep=1mm,
					xrightmargin=1pt,
					numbers=none,
					autogobble=true,
					escapeinside=||,
					formatcom=\color{codeblue}
					]{text}
					void Try(int k){
						for (int v = 1; v <= n; v++){
							|{\color{HUSTRed}        if (used[v] == 0)\{}|
							x[k] = v;
							|{\color{HUSTRed}            used[v] = 1;}|
							if (k == n) solution();
							else Try(k + 1);
							|{\color{HUSTRed}            used[v] = 0;}|
						}
					}
				}
				
				int main(){
					scanf("%d", &n);
					for (int v = 1; v <= n; v++) used[v] = 0;
					Try(1);
				}
			\end{minted}
		\end{minipage}
	};
	
	%--- Mũi tên nối hai khung ---
	\coordinate (Astart) at ([yshift=-0.2cm,xshift=0.1cm]L.east);
	\coordinate (Aend)   at ([yshift=-0.04cm,xshift=-0.1cm]R.west);
	\draw[->] (Astart) -- (Aend);
\end{tikzpicture}
\end{frame}


\begin{frame}[t,fragile]{BÀI TOÁN LIỆT KÊ HOÁN VỊ}
	\small
	\setlength{\columnsep}{0.1cm} % khoảng cách giữa 2 cột
	
	\begin{columns}[T,onlytextwidth]
		
		%=========== Cột trái ===========
		\begin{column}{0.40\textwidth}
			\raggedright
			% đẩy khung code vào trong một chút
			\hspace*{0.2\textwidth}%
			\begin{minipage}[t]{\dimexpr\linewidth-0.05\textwidth\relax}
				\begin{minted}[
					fontsize=\scriptsize,
					frame=single,
					framesep=2mm,
					xleftmargin=-10pt,
					numbers=none,
					autogobble=true,
					baselinestretch=1.5,
					formatcom=\color{HUSTBlue}
					]{text}
					#include <stdio.h>
					int n;
					int x[100];
					int used[100];
					
					void solution(){
						for (int k = 1; k <= n; k++)
						printf("%d ", x[k]);
						printf("\n");
					}
				\end{minted}
			\end{minipage}
		\end{column}
		
		%=========== Cột phải ===========
		\begin{column}{0.45\textwidth}
			\raggedright
			% đẩy khung code vào trong theo chiều ngược lại
			\hspace*{-0.1\textwidth}%
			\begin{minipage}[t]{\dimexpr\linewidth+0.05\textwidth\relax}
				\begin{minted}[
					fontsize=\scriptsize,
					frame=single,
					framesep=2mm,
					xleftmargin=-10pt,
					numbers=none,
					autogobble=true,
					escapeinside=||,
					formatcom=\color{HUSTBlue}
					]{text}
					void Try(int k){
						for (int v = 1; v <= n; v++){
							if (used[v] == 0){
								x[k] = v;
								|{\color{HUSTRed}       used[v] = 1;}|
								
								if (k == n) solution();
								else Try(k + 1);
								
								|{\color{HUSTRed}       used[v] = 0;}|
							}
						}
					}
					
					int main(){
						scanf("%d", &n);
						for (int v = 1; v <= n; v++) used[v] = 0;
						Try(1);
					}
				\end{minted}
			\end{minipage}
		\end{column}
		
	\end{columns}
\end{frame}

\section{Bài toán điền số Sudoku}
\begin{frame}[t]{BÀI TOÁN ĐIỀN SỐ SUDOKU}
	\begin{columns}[T,onlytextwidth]
		
		%================= Cột trái: chữ =================
		\begin{column}{0.52\textwidth}
			\small
			\setlength{\leftmargini}{-1.2em}
			
			\begin{itemize}
				\item Phát biểu bài toán:
				\begin{itemize}
					\item Cho lưới ô vuông $9\times 9$ được chia thành
					$9$ lưới $3\times 3$.
					\item Các hàng, cột được đánh số $0, 1, 2, \ldots, 8$.
					\item Liệt kê tất cả các cách điền các số $1,2,\ldots,9$
					vào các ô thuộc lưới ô vuông $9\times 9$ sao cho:
					\begin{itemize}
						\item Các số trên mỗi dòng là khác nhau,
						\item Các số trên mỗi cột là khác nhau,
						\item Các số trên mỗi lưới $3\times 3$ là khác nhau.
					\end{itemize}
				\end{itemize}
			\end{itemize}
		\end{column}
		
		%================= Cột phải: hình =================
		\begin{column}{0.48\textwidth}
			\centering
			\SudokuIndexedGrid[0.55]
			
		\end{column}
		
	\end{columns}
\end{frame}

\begin{frame}[t]{BÀI TOÁN ĐIỀN SỐ SUDOKU}
	\begin{columns}[T,onlytextwidth]
		
		%================= Cột trái: chữ =================
		\begin{column}{0.52\textwidth}
			\small
			\setlength{\leftmargini}{-1.2em}
			
			\begin{itemize}
				\item Phát biểu bài toán:
				\begin{itemize}
					\item Cho lưới ô vuông $9\times 9$ được chia thành
					$9$ lưới $3\times 3$.
					\item Các hàng, cột được đánh số $0, 1, 2, \ldots, 8$.
					\item Liệt kê tất cả các cách điền các số $1,2,\ldots,9$
					vào các ô thuộc lưới ô vuông $9\times 9$ sao cho:
					\begin{itemize}
						\item Các số trên mỗi dòng là khác nhau,
						\item Các số trên mỗi cột là khác nhau,
						\item Các số trên mỗi lưới $3\times 3$ là khác nhau.
					\end{itemize}
				\end{itemize}
			\end{itemize}
		\end{column}
		
		%================= Cột phải: hình nghiệm Sudoku =================
		\begin{column}{0.48\textwidth}
			\centering
			\SudokuSolvedGrid[0.55]{%
				{1,2,3,4,5,6,7,8,9},
				{4,5,6,7,8,9,1,2,3},
				{7,8,9,1,2,3,4,5,6},
				{2,1,4,3,6,5,8,9,7},
				{3,6,5,8,9,7,2,1,4},
				{8,9,7,2,1,4,3,6,5},
				{5,3,1,6,4,2,9,7,8},
				{6,4,2,9,7,8,5,3,1},
				{9,7,8,5,3,1,6,4,2},
			}
			
			
		\end{column}
	\end{columns}
\end{frame}

\begin{frame}[t]{BÀI TOÁN ĐIỀN SỐ SUDOKU}
	\begin{columns}[T,onlytextwidth]
		
		%================= Cột trái: chữ =================
		\begin{column}{0.58\textwidth}
			\small
			\setlength{\leftmargini}{-1.2em}
			
			\begin{itemize}
				\item Biểu diễn lời giải: $X[i,j]$ là giá trị số được điền 
				vào ô hàng $i$ cột $j$ ($i,j = 0,1,2,\ldots,8$).
				
				\item Mảng đánh dấu
				\begin{itemize}
					\item $\texttt{markR}[r,v] = 1$: giá trị $v$ đã xuất hiện trên hàng $r$,
					$\texttt{markR}[r,v] = 0$ ngược lại ($r = 0,1,\ldots,8$; $v = 1,2,\ldots,9$).
					
					\item $\texttt{markC}[c,v] = 1$: giá trị $v$ đã xuất hiện trên cột $c$,
					$\texttt{markC}[c,v] = 0$ ngược lại ($c = 0,1,\ldots,8$; $v = 1,2,\ldots,9$).
					
					\item $\texttt{markS}[i,j,v] = 1$: giá trị $v$ đã xuất hiện trong lưới $3\times 3$ hàng thứ $i$ và cột thứ $j$, 
					$\texttt{markS}[i,j,v] = 0$ ngược lại ($i,j = 0,1,2$; $v = 1,2,\ldots,9$).
				\end{itemize}
			\end{itemize}
		\end{column}
		
		%================= Cột phải: hình =================
		\begin{column}{0.42\textwidth}
			\centering
			\SudokuSolvedGrid[0.55]{%
				{1,2,3,4,5,6,7,8,9},
				{4,5,6,7,8,9,1,2,3},
				{7,8,9,1,2,3,4,5,6},
				{2,1,4,3,6,5,8,9,7},
				{3,6,5,8,9,7,2,1,4},
				{8,9,7,2,1,4,3,6,5},
				{5,3,1,6,4,2,9,7,8},
				{6,4,2,9,7,8,5,3,1},
				{9,7,8,5,3,1,6,4,2},
			}
		\end{column}
		
	\end{columns}
\end{frame}

\begin{frame}[t,fragile]{BÀI TOÁN ĐIỀN SỐ SUDOKU}
	\small
	\setlength{\leftmargini}{-1.2em}
	
	\begin{columns}[T,onlytextwidth]
		%================= Cột trái: bullet + check(v,r,c) =================
		\begin{column}{0.45\textwidth}
			\setlength{\leftmargini}{-1.2em}
			\begin{itemize}
				\item Thứ tự duyệt các ô để thử giá trị:
				từ trên xuống dưới và từ trái qua phải.
				\item Hàm đệ quy \texttt{Try(r, c)}: thử giá trị
				cho ô hàng \texttt{r} cột \texttt{c}.
			\end{itemize}
			
			\vspace{0.2cm}
			
			\begin{minipage}[t]{\linewidth}
				\begin{minted}[
					fontsize=\scriptsize,
					frame=single,
					rulecolor=\color{black},
					framesep=1mm,
					xleftmargin=-10pt,
					numbers=none,
					autogobble=true,
					baselinestretch=1.4
					]{text}
					check(v, r, c){
						if markR[r,v] = 1 then return 0;
						if markC[c,v] = 1 then return 0;
						if markS[r/3,c/3,v] = 1 then return 0;
						return 1;
					}
				\end{minted}
			\end{minipage}
		\end{column}
		
		%================= Cột phải: Try(r,c) =================
		\begin{column}{0.45\textwidth}
			\begin{minipage}[t]{\linewidth}
				\begin{minted}[
					fontsize=\scriptsize,
					frame=single,
					rulecolor=\color{black},
					framesep=1mm,
					xrightmargin=-10pt,
					numbers=none,
					autogobble=true,
					baselinestretch=1.0
					]{text}
					Try(r, c){
						for v = 1 to 9 do {
							if (check(v,r,c)) then {
								X[r,c] = v;
								markR[r,v] = 1;
								markC[c,v] = 1;
								markS[r/3,c/3,v] = 1;
								
								if r = 8 and c = 8 then solution();
								else {
									if c = 8 then Try(r+1, 0);
									else Try(r, c+1);
								}
								
								markR[r,v] = 0;
								markC[c,v] = 0;
								markS[r/3,c/3,v] = 0;
							}
						}
					}
				\end{minted}
			\end{minipage}
		\end{column}
	\end{columns}
\end{frame}

%Hết

{\HUSTUseBackground{theme_hust_oneside.pdf}
\begin{frame}
  \ifdefstring{\insertaspectratio}{169}{
    \placecontent{0.355\paperwidth}{0.410\paperheight}{0.640\paperwidth}{
        \color{HUSTRed}\bfseries\fontsize{28pt}{36pt}\selectfont\centering
        THANK YOU!
    }
  }{}
  \ifdefstring{\insertaspectratio}{43}{
    \placecontent{0.355\paperwidth}{0.440\paperheight}{0.640\paperwidth}{
        \color{HUSTRed}\bfseries\fontsize{28pt}{36pt}\selectfont\centering
        THANK YOU!
    }
  }{}
\end{frame}
}



\end{document}